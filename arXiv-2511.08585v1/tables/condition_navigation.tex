\begin{table*}[t]
\resizebox{\textwidth}{!}{%
\centering
\small
\setlength{\tabcolsep}{4.5pt} 
\begin{tabular}{c|ccccccc|ccc}
\toprule
\multirow{2}{*}{Characteristics} & \multicolumn{7}{c|}{Spatial Condition} & \multicolumn{3}{c}{Navigation Mode} \\


& HD Map & Layout & Text Description & Canny & Depth & Sketch & Motion Pose & Trajectory & Action & Text Instruction\\
\midrule
Temporality & \xmark & \xmark & \xmark & \cmark & \cmark & \cmark & \cmark & \cmark & \cmark & \cmark \\

Content Independence & \xmark & \xmark & \cmark & \xmark & \xmark & \xmark & \xmark & \cmark & \cmark & \cmark \\

Spatial-Reasoning & \cmark & \cmark & \xmark & \xmark & \cmark & \cmark & \xmark & \cmark & \cmark & \cmark \\

\bottomrule
\end{tabular}
}
\vspace{0.1cm}
\caption{\textbf{Comparison Between Spatial Conditions and Navigation Modes.} In this table, we compare the key terms “spatial condition” and “navigation mode”, focusing on their main characteristics and differences. The comparison is conducted primarily across three dimensions: temporality, content independence, and spatial reasoning.}
% \vspace{-0.3cm}
\label{tab:condition_navigation_comparison}
\end{table*}

